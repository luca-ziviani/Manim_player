\documentclass[a4paper]{article}

%   PACKAGES

\usepackage[utf8]{inputenc}
\usepackage{mathtools}
\usepackage{amsmath, amsthm, amsfonts,amssymb}
\usepackage{enumitem}
\usepackage{graphicx}
\usepackage{colortbl}
\usepackage{tikz}
\usepackage{esint}
\usepackage{mathrsfs}
\usepackage{subfig,float}
%---------------------------------------
%\usepackage[T1]{fontenc} 
%---------------------------------------
\usepackage{bbm} 
\usepackage{enumitem}
\usepackage{mathtools}
\usepackage{dsfont}
\usepackage{xcolor}
\usepackage{csquotes}
\usepackage[colorlinks=true,linkcolor=blue,citecolor=blue,urlcolor=blue,breaklinks]{hyperref}
\usepackage{cleveref}
\usepackage[square,sort,comma,numbers]{natbib}
\usepackage{url}

\setlength{\textwidth}{12cm}
\setlength{\linewidth}{12cm}
\setlength{\hsize}{12cm}

\newcommand{\pause}{}

\newcommand{\set}[1]{\left\{#1\right\}}
\newcommand{\M}{\mathcal{M}}
\newcommand{\cL}{\mathcal{L}}
\newcommand{\cA}{\mathcal{A}}
\newcommand{\cB}{\mathcal{B}}
\newcommand{\cI}{\mathcal{I}}
\newcommand{\cH}{\mathcal{H}}
\newcommand{\bI}{\mathbb{I}}
\newcommand{\sfL}{\mathsf{L}}
\newcommand{\sfT}{\mathsf{T}}
\newcommand{\sfH}{\mathsf{H}}
\newcommand{\sfA}{\mathsf{A}}
\newcommand{\sfD}{\mathsf{D}}
\newcommand{\sign}{\text{sgn}}
\newcommand{\bangle}[1]{\langle #1\rangle}
\newcommand{\wangle}[1]{\lfloor #1 \rceil}
\newcommand{\wx}{\lfloor x \rceil}
\newcommand{\wy}{\lfloor y \rceil} 

\renewcommand{\(}{\left(}
\renewcommand{\)}{\right)}

\DeclareMathOperator{\supp}{supp}
\DeclareMathOperator{\essup}{ess\,sup}
\DeclareMathOperator{\Lip}{Lip}
\DeclareMathOperator{\sgn}{sgn}
\DeclareMathOperator{\hess}{Hess}
\DeclareMathOperator{\Div}{div}

\DeclarePairedDelimiter\abs{\lvert}{\rvert}
\DeclarePairedDelimiter\norm{\lVert}{\rVert}


\def\d{\,\mathrm{d}}
\def\dv{\d v}
\def\dx{\d x}
\def\dt{\partial_t}
\def\ddt{\frac{\mathrm{d}}{\mathrm{d}t}}
\def\grad{\nabla}
\def\gradx{\nabla_x}
\def\gradv{\nabla_v}
\def\R{\mathbb{R}}
\def\C{\mathbb{C}}
\def\S{\mathbb{S}}
\def\P{\mathcal{P}}
\def\NN{\mathbb{N}}
\def\Q{\mathbb{Q}}
\def\1{\mathds{1}}
\def\RRd{ {\R^d\times\R^d} }
\def\Rd{ {\R^d} }



%-----------------------------------------------------------------
\newtheorem{thm}{Theorem}[section]
\newtheorem{cor}[thm]{Corollary}
\newtheorem{lem}[thm]{Lemma}
\newtheorem{prop}[thm]{Proposition}
\newtheorem{claim}[thm]{Claim}
\newtheorem{hypothesis}[thm]{Hypothesis}

\newtheorem{asm}{Assumption}
\theoremstyle{definition}
\newtheorem{hyp}{Hypothesis}
\newtheorem{dfn}[thm]{Definition}
\theoremstyle{remark}
\newtheorem{remark}[thm]{Remark}
\newtheorem{ex}[thm]{Example}



\begin{document}

\section{Stein-Weiss interpolation theorem}

Preliminaries:

$\bullet$ Let $w\colon \Rd\to\R$ a positive function. \pause For every function $f\colon \Rd\to \R$ we denote the \textbf{weighted $L^p$-norm of $f$} 
\[
\norm{f}_{L^p(w)} = \left(\int_\Rd \abs{f(x)}^p w(x)dx \right)^{\frac1p}.
\]
$\bullet$ We consider the space
\[
L^p(w) = \left\{ f\colon \Rd\to \R \Big\vert \norm{f}_{L^p(w)} <\infty \right\}.
\]
composed by all the functions with bounded norm. 

$\bullet$ We say that $T\colon L^p(w )\to L^p(w)$ is \textbf{linear} if for any $f,g\in L^p(w )$ and any $\alpha,\beta\in\R$ we have
\[
T(\alpha f+ \beta g) = \alpha Tf + \beta Tg.
\]
Moreover, $T$ is \textbf{bounded} if there exists a constant $M>0$ such that
\[
\norm{Tf}_{L^p(w)} \leq M \norm{f}_{L^p( w)}
\]
for any function $f\in L^p( w)$.

Question:\pause If a linear operator $T$ is bounded on two spaces
\begin{align*}
T& \colon L^p( w_0)\to L^p(w_0) &  T& \colon L^p( w_1 )\to L^p(w_1)
\end{align*}
for some different weights $w_0$ and $w_1$,\pause is it bounded even for some "intermediate" weight $w$ between $w_0$ and $w_1$ ?

\begin{thm}[Stein-Weiss]
Assume that $1\leq p< \infty$ and that $0<\theta<1$. \pause Let $w_0$, $w_1$ be positive weight functions and suppose that $T$ is a bounded linear operator such that
\begin{align*}
T\pause& \colon L^p( w_0)\to L^p(w_0) \pause&  T\pause& \colon L^p( w_1 )\to L^p(w_1)
\end{align*}
with norms $M_0$ and $M_1$ respectively. \pause Then 
\begin{align*}
T\colon L^p(w )\to L^p(w)
\end{align*}
with norm $M\leq M_0^{1-\theta} M_1^\theta$ and 
\begin{align*}
w & = w_0^{1-\theta}w_1^{\theta} .
\end{align*}
\end{thm}

\begin{proof}
Here below we reproduce a general interpolation method, called \textbf{K interpolation method}.\pause For this particular theorem, it reduces to a very nice proof with lovely computations on integrals.

Let $f$ be a function, suppose that you can decompose it as a sum of two other functions $f_0,f_1$, \pause that is
\[
f= f_0+f_1.
\]
Then, for any $t\geq0$ fixed, the quantity
\[
 \norm{f_0}^p_{L^p(w_0)} + t^p\norm{f_1}^p_{L^p(w_1)}\geq 0
\]
is always positive (or infinity).\pause Notice that we took the $L^p$ norm of $f_0$ with weight $w_0$\pause and the $L^p$ norm of $f_1$ with weight $w_1$.\pause Of course, this quantity depends on $f$ but it also depend on the particular decomposition $f=f_0+f_1$ we chose.\pause Notice that, if $f\neq 0$, then we always obtain a positive number.\pause  We may ask what is the smallest value we can get over all possible decompositions of $f$. \pause For this purpose, we define the functional 
\[
K_p(t,f) = \inf_{f=f_0+f_1}( \norm{f_0}^p_{L^p(w_0)} + t^p\norm{f_1}^p_{L^p(w_1)})^{\tfrac{1}{p}}.
\]
Notice that $K_p(t,f)$ is a positive number (or infinity) for any fixed $t\geq0$ and function $f$.\pause How can we use it? \pause The winning idea is to look at the following integral
\[
\Phi(f) := \left(\int_0^{\infty} (t^{-\theta} K_p(t,f) )^p  \tfrac{dt}{t}\right)^{\tfrac{1}{q}},
\]
where $0<\theta<1$. \pause We are integrating with respect to $t$, so at the end we obtain a number which only depends on $f$.\pause We claim that this defines a norm.\pause To understand why, notice that
\begin{align*}
K_p(t,f) \pause&= \inf_{f=f_0+f_1} \left( \int_\Rd\abs{f_0(x)}^p w_0(x)dx + t^p\int_\Rd\abs{f_1(x)}^pw_1(x)dx\right)^{\tfrac{1}{p}}\\
\pause& = \left( \int_\Rd\inf_{f=f_0+f_1}\left(\abs{f_0(x)}^p w_0(x) + t^p\abs{f_1(x)}^pw_1(x)\right)dx\right)^{\tfrac{1}{p}}\\
\pause& = \left( \int_\Rd|f(x)|^p\inf_{y_0+y_1=1}\left(\abs{y_0}^p w_0(x) + t^p\abs{y_1}^pw_1(x)\right)dx\right)^{\tfrac{1}{p}}\\
\pause& = \left( \int_\Rd|f(x)|^p w_0(x)\inf_{y_0+y_1=1}\left(\abs{y_0}^p  + t^p\tfrac{w_1(x)}{w_0(x)}\abs{y_1}^p\right)dx\right)^{\tfrac{1}{p}}
\end{align*}
For clarity, we define
\[
F(s) =\inf_{y_0+y_1=1}\left(\abs{y_0}^p  + s\abs{y_1}^p\right),
\]
so we can rewrite
\[
K_p(t,f) = \left( \int_\Rd|f(x)|^p w_0(x) F\left(t^p\tfrac{w_1(x)}{w_0(x)}\right) dx\right)^{\tfrac{1}{p}}\\
\]
Now we can substitute in the definition of $\Phi$ and we obtain
\begin{align*}
\Phi(f) \pause& = \left( \int_0^\infty t^{-\theta p} (K_p(t,f))^p \tfrac{dt}{t} \right)^{\tfrac{1}{p}} \\
\pause& = \left( \int_0^\infty t^{-\theta p} \left( \int_\Rd|f(x)|^p w_0(x) F\left(t^p\tfrac{w_1(x)}{w_0(x)}\right) dx\right) \tfrac{dt}{t} \right)^{\tfrac{1}{p}} \\
\pause& = \left( \int_\Rd|f(x)|^p w_0(x) \left(\int_0^\infty t^{-\theta p}F\left(t^p\tfrac{w_1(x)}{w_0(x)}\right)\tfrac{dt}{t} \right) dx  \right)^{\tfrac{1}{p}}.
\end{align*}
The change of variable $s=t\left(\frac{w_1(x)}{w_0(x)}\right)^{\frac{1}{p}}$ gives
\begin{align*}
\Phi(f)\pause& = \left( \int_\Rd|f(x)|^p w_0^{1-\theta}(x)w_1^{\theta}(x) \left(\int_0^\infty s^{-\theta p}F\left(s^p\right)\tfrac{ds}{s} \right) dx  \right)^{\tfrac{1}{p}} \\
\pause& = C \left( \int_\Rd|f(x)|^p w_0^{1-\theta}(x)w_1^{\theta}(x)  dx  \right)^{\tfrac{1}{p}}\\
\pause&  = C\norm{f}_{L^p(w)}
\end{align*}
where
\[
C = \int_0^\infty s^{-\theta p}F\left(s^p\right)\tfrac{ds}{s} <\infty
\]
This proves that $\Phi$ is actually the norm with weight $w$ !\pause We built the bridge between the norm with the new weight $w$ by just using the norms with the two original weight $w_0, w_1$. \pause Now we can conclude the proof of the theorem. \pause We have
\begin{align*}
K_p(t, Tf) \pause& = \inf_{f=f_0+f_1}( \norm{Tf_0}^p_{L^p(w_0)} + t^p\norm{Tf_1}^p_{L^p(w_1)})^{\tfrac{1}{p}}\\ 
	\pause& \leq \inf_{f=f_0+f_1}(M_0^p \norm{f_0}^p_{L^p(w_0)} + t^p \,M_1^p\norm{f_1}^p_{L^p(w_1)})^{\tfrac{1}{p}}\\
	\pause& \leq M_0 \inf_{f=f_0+f_1}( \norm{f_0}^p_{L^p(w_0)} + t^p \,\tfrac{M_1^p}{M_0^p}\norm{f_1}^p_{L^p(w_1)})^{\tfrac{1}{p}} \\
	\pause&  \leq M_0 K_p\left( \tfrac{M_1}{M_0}t,\, f \right).
\end{align*}
By applying $\Phi$ and changing variables as before, we obtain
\begin{align*}
\norm{Tf}_{L^p(w)} \pause& = \Phi (Tf) \\
\pause & \leq M_0\left( \int_0^\infty t^{-\theta p} K_p\left( \tfrac{M_1}{M_0}t,\, f \right)^p \tfrac{dt}{t} \right)^{\tfrac{1}{p}} \\
%	\pause& \leq M_0 \Phi_{\theta,q} \left(K_p\left(\tfrac{M_1}{M_0}t, \,f\right)\right) \\
\pause& \leq M_0^{1-\theta}M_1^\theta  \left( \int_0^\infty t^{-\theta p} (K_p(t,f))^p \tfrac{dt}{t} \right)^{\tfrac{1}{p}}  \\
\pause& \leq M_0^{1-\theta}M_1^\theta\Phi(f)\\
\pause& \leq M_0^{1-\theta}M_1^\theta \norm{f}_{L^p(w)}.
\end{align*}
This proves that $T$ is bounded on $L^p(\Rd,w)$ with norm $M=M_0^{1-\theta} M_1^{\theta}$.
\end{proof}



\end{document}



