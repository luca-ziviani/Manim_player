\documentclass[a4paper,11pt]{amsart}

%   PACKAGES

\usepackage[utf8]{inputenc}
\usepackage{mathtools}
\usepackage{amsmath, amsthm, amsfonts,amssymb}
\usepackage{enumitem}
\usepackage{graphicx}
\usepackage{colortbl}
\usepackage{tikz}
\usepackage{esint}
\usepackage{mathrsfs}
\usepackage{subfig,float}
%---------------------------------------
%\usepackage[T1]{fontenc} 
%---------------------------------------
\usepackage{bbm} 
\usepackage{enumitem}
\usepackage{mathtools}
\usepackage{dsfont}
\usepackage{xcolor}
\usepackage{csquotes}
\usepackage[colorlinks=true,linkcolor=blue,citecolor=blue,urlcolor=blue,breaklinks]{hyperref}
\usepackage{cleveref}
\usepackage[square,sort,comma,numbers]{natbib}
\usepackage{url}


\newcommand{\set}[1]{\left\{#1\right\}}
\newcommand{\M}{\mathcal{M}}
\newcommand{\cL}{\mathcal{L}}
\newcommand{\cA}{\mathcal{A}}
\newcommand{\cB}{\mathcal{B}}
\newcommand{\cI}{\mathcal{I}}
\newcommand{\cH}{\mathcal{H}}
\newcommand{\bI}{\mathbb{I}}
\newcommand{\sfL}{\mathsf{L}}
\newcommand{\sfT}{\mathsf{T}}
\newcommand{\sfH}{\mathsf{H}}
\newcommand{\sfA}{\mathsf{A}}
\newcommand{\sfD}{\mathsf{D}}
\newcommand{\sign}{\text{sgn}}
\newcommand{\bangle}[1]{\langle #1\rangle}
\newcommand{\wangle}[1]{\lfloor #1 \rceil}
\newcommand{\wx}{\lfloor x \rceil}
\newcommand{\wy}{\lfloor y \rceil} 

\renewcommand{\(}{\left(}
\renewcommand{\)}{\right)}

\DeclareMathOperator{\supp}{supp}
\DeclareMathOperator{\essup}{ess\,sup}
\DeclareMathOperator{\Lip}{Lip}
\DeclareMathOperator{\sgn}{sgn}
\DeclareMathOperator{\hess}{Hess}
\DeclareMathOperator{\Div}{div}

\DeclarePairedDelimiter\abs{\lvert}{\rvert}
\DeclarePairedDelimiter\norm{\lVert}{\rVert}


\def\d{\,\mathrm{d}}
\def\dv{\d v}
\def\dx{\d x}
\def\dt{\partial_t}
\def\ddt{\frac{\mathrm{d}}{\mathrm{d}t}}
\def\grad{\nabla}
\def\gradx{\nabla_x}
\def\gradv{\nabla_v}
\def\R{\mathbb{R}}
\def\C{\mathbb{C}}
\def\S{\mathbb{S}}
\def\P{\mathcal{P}}
\def\NN{\mathbb{N}}
\def\Q{\mathbb{Q}}
\def\1{\mathds{1}}
\def\RRd{ {\R^d\times\R^d} }
\def\Rd{ {\R^d} }



%-----------------------------------------------------------------
\newtheorem{thm}{Theorem}[section]
\newtheorem{cor}[thm]{Corollary}
\newtheorem{lem}[thm]{Lemma}
\newtheorem{prop}[thm]{Proposition}
\newtheorem{claim}[thm]{Claim}
\newtheorem{hypothesis}[thm]{Hypothesis}

\newtheorem{asm}{Assumption}
\theoremstyle{definition}
\newtheorem{hyp}{Hypothesis}
\newtheorem{dfn}[thm]{Definition}
\theoremstyle{remark}
\newtheorem{remark}[thm]{Remark}
\newtheorem{ex}[thm]{Example}


\begin{document}

\section{Stein-Weiss interpolation theorem}

\begin{thm}
Assume that $1\leq p\leq \infty$ and that $0<\theta<1$. Let $w_0$, $w_1$, $\Tilde{w}_0$, $\Tilde{w}_1$ be positive weight functions and consider 
\begin{align*}
w &= w_0^{1-\theta}w_1^{\theta} & \Tilde{w} &= \Tilde{w}_0^{1-\theta}\Tilde{w}_1^{\theta}.
\end{align*}
If $T$ is a bounded linear operator such that
\begin{align*}
T&\colon L^p(U, w_0d\mu )\to L^p(V, \Tilde{w}_0d\nu) & T&\colon L^p(U, w_1d\mu )\to L^p(V, \Tilde{w}_1d\nu)
\end{align*}
with norms $M_0$ and $M_1$ respectively, then
\begin{align*}
T\colon L^p(U, wd\mu )\to L^p(V, \Tilde{w}d\nu)
\end{align*}
with norm $M\leq M_0^{1-\theta} M_1^\theta$.
\end{thm}


\begin{proof}
Consider the functional 
\[
K_p(t,f) = \inf_{f=f_0+f_1}( \norm{f_0}^p_{L^p(w_0)} + t^p\norm{f_1}^p_{L^p(w_1)})^{\tfrac{1}{p}}.
\]
The functional 
\[
f\mapsto \Phi_{\theta,q} (K_p(t,f)) = \left(\int_0^{\infty} (t^{-\theta} K_p(t,f) )^q  \tfrac{dt}{t}\right)^{\tfrac{1}{q}}
\]
satisfies all the properties of a norm.To understand what this norm is, notice that
\begin{align*}
K_p(t,f) &= \inf_{f=f_0+f_1}\left( \int_U\abs{f_0(x)}^p w_0(x)d\mu(x) + t^p\int_U\abs{f_1(x)}^pw_1(x)d\mu(x)\right)^{\tfrac{1}{p}}\\
&= \left( \int_U\inf_{f=f_0+f_1}\left(\abs{f_0(x)}^p w_0(x) + t^p\abs{f_1(x)}^pw_1(x)\right)d\mu(x)\right)^{\tfrac{1}{p}}\\
&= \left( \int_U|f(x)|^p\inf_{y_0+y_1=1}\left(\abs{y_0}^p w_0(x) + t^p\abs{y_1}^pw_1(x)\right)d\mu(x)\right)^{\tfrac{1}{p}}\\
&= \left( \int_U|f(x)|^p w_0(x)\inf_{y_0+y_1=1}\left(\abs{y_0}^p  + t^p\tfrac{w_1(x)}{w_0(x)}\abs{y_1}^p\right)d\mu(x)\right)^{\tfrac{1}{p}}\\
&= \left( \int_U|f(x)|^p w_0(x) F\left(t^p\tfrac{w_1(x)}{w_0(x)}\right) d\mu(x)\right)^{\tfrac{1}{p}}\\
\end{align*}
where
\[
F(s) =\inf_{y_0+y_1=1}\left(\abs{y_0}^p  + s\abs{y_1}^p\right)
\]
By the Fubini theorem we have
\begin{align*}
\Phi_{\theta,p}(K_p(t,f)) &= \left( \int_0^\infty t^{-\theta p} (K_p(t,f))^p \tfrac{dt}{t} \right)^{\tfrac{1}{p}} \\
&= \left( \int_0^\infty t^{-\theta p} \left( \int_U|f(x)|^p w_0(x) F\left(t^p\tfrac{w_1(x)}{w_0(x)}\right) d\mu(x)\right) \tfrac{dt}{t} \right)^{\tfrac{1}{p}} \\
&= \left( \int_U|f(x)|^p w_0(x) \left(\int_0^\infty t^{-\theta p}F\left(t^p\tfrac{w_1(x)}{w_0(x)}\right)\tfrac{dt}{t} \right) d\mu(x)  \right)^{\tfrac{1}{p}} \\
&= \left( \int_U|f(x)|^p w_0^{1-\theta}(x)w_1^{\theta}(x) \left(\int_0^\infty t^{-\theta p}F\left(s^p\right)\tfrac{dt}{t} \right) d\mu(x)  \right)^{\tfrac{1}{p}} \\
&= C \left( \int_U|f(x)|^p w_0^{1-\theta}(x)w_1^{\theta}(x)  d\mu(x)  \right)^{\tfrac{1}{p}}\\
& = C\norm{f}_{L^p(w)}
\end{align*}
where
\[
C = \int_0^\infty t^{-\theta p}F\left(s^p\right)\tfrac{dt}{t} <\infty
\]
We have
\begin{align*}
K_p(t, Tf) &= \inf_{f=f_0+f_1}( \norm{Tf_0}^p_{L^p(w_0)} + t^p\norm{Tf_1}^p_{L^p(w_1)})^{\tfrac{1}{p}}\\ 
	&\leq \inf_{f=f_0+f_1}(M_0^p \norm{f_0}^p_{L^p(w_0)} + t^p \,M_1^p\norm{f_1}^p_{L^p(w_1)})^{\tfrac{1}{p}}\\
	&\leq M_0 \inf_{f=f_0+f_1}( \norm{f_0}^p_{L^p(w_0)} + t^p \,\tfrac{M_1^p}{M_0^p}\norm{f_1}^p_{L^p(w_1)})^{\tfrac{1}{p}} \\
	& \leq M_0 K_p\left( \tfrac{M_1}{M_0}t,\, f \right).
\end{align*}
By applying $\Phi_{\theta,q}$ we obtain
\begin{align*}
\norm{Tf}_{L^p(w)} &= \Phi_{\theta,q} (K_p(t,Tf)) \\
	&\leq M_0 \Phi_{\theta,q} \left(K_p\left(\tfrac{M_1}{M_0}t, \,f\right)\right) \\
&\leq M_0^{1-\theta}M_1^\theta \Phi_{\theta,q} \left(K_p\left(t, \,f\right)\right) \\
&\leq M_0^{1-\theta}M_1^\theta \norm{f}_{L^p(w)}.
\end{align*}




\end{proof}


\end{document}


\begin{align*}
    \mathcal{M}(v)& =\frac{1}{(2\pi)^{d/2}}e^{-\frac{|v|^2}{2}} &  \rho_f(x)&=\int_\Rd f(x,v)dv.
\end{align*}

%\end{document}

\section{DMS method for BGK with explicit macroscopic quantities}

Consider the equation
\begin{equation}\label{eq:BGK}
    \partial_t f + v\cdot\nabla_x f -\nabla_x V\cdot\nabla_v f = \M\rho_f-f
\end{equation}
where
\begin{align*}
    \mathcal{M}(v)& =\frac{1}{(2\pi)^{d/2}}e^{-\frac{|v|^2}{2}} &  \rho_f(x)=\int_\Rd f(x,v)dv.
\end{align*}
There exists one unique steady state 
\begin{equation}
    G(x,v)=e^{-V(x)}\mathcal{M}(v)
\end{equation}
and we consider the Hilbert space $\mathcal{H}:=L^2(G^{-1}dxdv)$ denoting as $\langle{\cdot,\cdot}\rangle$ and $\|{\cdot}\|$ its scaar product and norm. The projection $\Pi \colon\mathcal{H}\to\mathcal{H}$ is defined as
\begin{equation}
    (\Pi f)(x,v)=\mathcal{M}(v)\rho_f(x)
\end{equation}
for all $f\in\mathcal{H}$. 



\subsubsection*{Microscopic coercivity} Let $f_t(x,v)$ be a solution to \eqref{eq:BGK} with initial datum $f_0=f_0(x,v)$, then we have
\begin{align*}
    \ddt\frac{\norm{f_t}_\cH^2}{2}&= \iint_\RRd (-v\cdot\gradx f_t + \gradx V\cdot\gradv f_t)f_t\frac{dxdv}{G} + \iint_\RRd (\M\rho_f-f)f\frac{dxdv}{G}\\
    &=\frac{1}{2}\iint_\RRd f_t^2\,\frac{v\cdot \gradx G-\gradx V\cdot \gradv G}{G^2}dxdv + \iint_\RRd (f_t\Pi f_t-f_t^2) \frac{dxdv}{G}\\
    &\leq \frac{1}{2}\iint_\RRd (-(\Pi f_t)^2 + 2f_t\Pi f_t - f_t^2)\frac{dxdv}{G}\\
    &=-\iint_\RRd \abs{f_t-\Pi f_t}^2\frac{dxdv}{G}\\
    &=-\norm{(1-\Pi)f_t}^2,
\end{align*}
where we used that
\begin{align*}
    \iint_\RRd (\Pi f_t)^2\frac{dxdv}{G}&=\iint_\RRd \rho_f(x)^2 \M(v)e^{V(x)}dxdv\\
    &=\int_\Rd\(\int_\Rd f(x,v)dv\)^2e^{V(x)}dx\\
    &\leq \int_\Rd\(\int_\Rd f^2(x,v)\frac{dv}{\M(v)}\)\(\int_\Rd\M(v)dv\)e^{V(x)}dx\\
    &=\iint_\RRd f_t^2\frac{dxdv}{G}.
\end{align*}
This estimates are not enough to conclude the convergence, so we need to add a perturbation term to $\norm{\cdot}$. 



\subsubsection*{Macroscopic quantities} We introduce the macroscopic quantities
\begin{align*}
    \rho_f(x)&=\int_\Rd f(x,v)dv & m_f(x)&=\int_\Rd v\, f(x,v)dv & P_f(x)&=\int_\Rd v\otimes v f(x,v)dv
\end{align*}
and we want to see if we can control some combination of them by the microscopic component $\norm{(1-\Pi)f}$. Notice that 
\begin{equation*}
    \int_\Rd v(\Pi f) dv = \rho_f\int_\Rd v\M(v)dv=0,
\end{equation*}
therefore we have
\begin{align*}
    \int_\Rd \abs{m_f}^2 e^{V(x)}dx&=\int_\Rd\( \int_\Rd v f(x,v)dv\)e^{V(x)}dx\\
    &=\int_\Rd\( \int_\Rd ((1-\Pi) f(x,v) )vdv\)^2e^{V(x)}dx\\
    &\leq \int_\Rd \(\int_\Rd \abs{(1-\Pi)f}^2\frac{dv}{\M(v)}\)\(\int_\Rd \abs{v}^2\M(v)dv\)e^{V(x)}dx\\
    &\leq d \norm{(1-\Pi)f}^2.
\end{align*}
We have used that
\[
\int_\Rd \abs{v}^2\M(v)dv = d.
\]
Similarly we have 
\begin{align*}
    \int_\Rd v\otimes v(\Pi f)dv= \rho_f\int_\Rd v\otimes v\M(v)dv = \rho_f \bI_d,
\end{align*}
thus if we define $E_f:=P_f-\rho_f\bI_d$ we have
\begin{align*}
    \int_\Rd\abs{E_f}^2e^{V(x)}dx&=\int_\Rd\(\int_\Rd v\otimes v \,(1-\Pi)f(x,v)dv\)^2e^{V(x)}dx\\
    &\leq \int_\Rd \(\int_\Rd \abs{(1-\Pi)f}^2\frac{dv}{\M(v)}\)\(\int_\Rd \abs{v\otimes v}^2\M(v)dv\)e^{V(x)}dx\\
    &\leq C_4 \norm{(1-\Pi)f}^2
\end{align*}
where $C_4 = \int_\Rd \abs{v\otimes v}^2\M(v)dv$.

Next we want to determine the evolution of $\rho_f$ and $m_f$ along the solutions of \eqref{eq:BGK}. It is enough to integrate \eqref{eq:BGK} and the same equation multiplied by $v$ to get
\begin{equation}
    \dt \rho_f + \Div_x m_f=0
\end{equation}
and
\begin{equation}
    \dt m_f + \Div_x P_f + \rho_f \, \gradx V = -m_f.
\end{equation}
The last equation can be rewritten as
\begin{equation}
    \dt m_f + \Div_x E_f +\grad_x\rho_f+ \rho_f \, \gradx V = -m_f.
\end{equation}
The crucial point is to recognise that the operator $A\colon\rho\mapsto -\grad_x\rho - \rho\gradx V$ is the adjoint of $\rho\mapsto \gradx \rho$ in $L^2(e^{V})$.

\subsubsection*{Hypocoercivity estimates} The idea of the DMS method is to introduce the entropy
\begin{equation}
    H(f):=\norm{f}^2 -\epsilon \bangle{B^{-1}(\Div_x m_f),\rho_f}_{\sfL^2(e^V)}
\end{equation}
where $B$ is the elliptic operator 
\[
Bu:= u -\Div_x(\gradx u+u\gradx V).
\]
Let us show that $H(f)$ is equivalent to $\norm{\cdot}^2$. Let $g=g(x)$ define as $g=B^{-1}\Div_x m_f$, in other words $g$ is the solution to the elliptic equation
\[
g-\Div_x(\gradx g +g\gradx V)=\Div_x m_f.
\]
Testing with $g\, e^V$ and integrating by part we have
\begin{align*}
    \int_\Rd g^2e^{V}dx+ \int_\Rd (\gradx g+g\gradx V)^2 e^{V}dx &=\int_\Rd g \Div_x m_f\, e^Vdx\\
    &=-\int_\Rd m_f(\gradx g+g\gradx V)e^Vdx\\
    &\leq \frac{1}{4}\int_\Rd \abs{m_f}^2 e^V dx +\int_\Rd (\gradx g+g\gradx V)^2e^V dx,
\end{align*}
which gives
\begin{equation*}
    \int_\Rd g^2 e^Vdx\leq \frac{1}{4}\int_\Rd \abs{m_f}^2 e^V.
\end{equation*}
But we can also deduce
\begin{align*}
    \int_\Rd (\gradx g+g\gradx V)^2 e^{V}dx&\leq\int_\Rd g^2e^{V}dx+ \int_\Rd (\gradx g+g\gradx V)^2 e^{V}dx\\
    &=-\int_\Rd m_f(\gradx g+g\gradx V)e^Vdx\\
    &\leq \frac{1}{2}\int_\Rd \abs{m_f}^2 e^V dx +\frac12\int_\Rd (\gradx g+g\gradx V)^2e^V dx,
\end{align*}
i.e.
\[
\int_\Rd (\gradx g+g\gradx V)^2 e^{V}dx\leq \frac12 \int_\Rd \abs{m_f}^2 e^V dx.
\]
Summarizing and expiciting $g$ we have 
\begin{equation}\label{eq:bdd1}
    \int_\Rd \abs{B^{-1}(\Div_x m_f)}^2e^Vdx\leq \frac14 \int_\Rd \abs{m_f}^2 e^V dx
\end{equation}
and
\begin{equation}\label{eq:bdd2}
    \int_\Rd \abs{\gradx\( (B^{-1}(\Div_x m_f))e^{V}\)}^2e^{-V}dx\leq \frac12 \int_\Rd \abs{m_f}^2 e^V dx
\end{equation}
In particular, thanks to \eqref{eq:bdd1}, we have
\begin{align*}
    \bangle{B^{-1}(\Div_x m_f),\rho_f}&\leq\(\int_\Rd\abs{m_f}^2e^Vdx\)^{1/2}\(\int_\Rd \rho_f^2e^Vdx\)^{1/2}\\
    &\leq \norm{(1-\Pi)f}_{L^2(G^{-1})}\norm{\Pi f}_{L^2(G^{-1})}\\
    &\leq \norm{f}^2_{L^2(G^{-1})}
\end{align*}
and we can finally conclude that $H(f)$ is equivalent to $\norm{\cdot}$. Now we loot at the dissipation of $H(f)$. We have
\begin{align*}
    -\ddt \bangle{B^{-1}(\gradx\cdot m_f, \rho_f)}_{\sfL^2(e^V)} =& -\bangle{B^{-1}(\Div_x\dt m_f), \rho_f}_{\sfL^2(e^V)}-\bangle{B^{-1}(\Div_x m_f), \dt\rho_f}_{\sfL^2(e^V)} \\
    =& \bangle{B^{-1}\Div_x(\Div_x E_f),\rho_f}_{\sfL^2(e^V)} + \bangle{B^{-1}\Div_x(\gradx\rho_f+\rho_f\gradx V),\rho_f}_{\sfL^2(e^V)}+\\
    &+\bangle{B^{-1}\Div_x m_f,\rho_f}_{\sfL^2(e^V)} +\bangle{B^{-1}\Div_x m_f,\Div_x m_f}_{\sfL^2(e^V)}
\end{align*}

\subsubsection{Macroscopic term} We can rewrite 
\[
\bangle{B^{-1}\Div_x(\gradx\rho_f+\rho_f\gradx V),\rho_f}_{\sfL^2(e^V)} = -\bangle{(1+A)^{-1}A\rho_f,\rho_f}_{\sfL^2(e^V)}.
\]
Observe that
\begin{align*}
    \bangle{A\rho_f,\rho_f}_{\sfL^2(e^V)}&=-\int_\Rd \Div_x(\gradx \rho_f+\rho_f\gradx V) \,\rho_f e^{V(x)}dx\\
    &=\int_\Rd (\gradx\rho_f+\rho_f\gradv V)^2 e^{V(x)}dx\\
    &=\int_\Rd \abs*{\gradx\(\frac{\rho_f}{e^{-V(x)}}\)}^2e^{-V(x)}dx\\
    &\geq C_P\int_\Rd \(\frac{\rho_f}{e^{-V(x)}}\)^2e^{-V(x)}dx\\
    &\geq C_P\norm{\Pi f}^2_{L^2(G^{-1})}
\end{align*}
and that the function $s\mapsto \frac{s}{1+s}$ is increasing for $s\geq 0$. As a consequence
\[
\bangle{(1+A)^{-1}A\rho_f,\rho_f}_{\sfL^2(e^V)}\geq \frac{C_P}{1+C_P}\norm{\Pi f}^2_{L^2(G^{-1})}
\]
and we conclude
\begin{equation}
    \boxed{\bangle{B^{-1}\Div_x(\gradx\rho_f+\rho_f\gradx V),\rho_f}_{\sfL^2(e^V)}\leq - \frac{C_P}{1+C_P}\norm{\Pi f}^2_{L^2(G^{-1})}}
\end{equation}

\subsubsection{Error terms} Thanks to \eqref{eq:bdd1} we have
\begin{align*}
    \bangle{B^{-1}\Div_x m_f , \rho_f}_{\sfL^2(e^V)}&\leq \(\frac14\int_\Rd\abs{m_f}^2e^Vdx\)^{1/2}\(\int_\Rd \rho_f^2e^Vdx\)^{1/2}\\
    &\leq \frac12\norm{(1-\Pi)f}_{L^2(G^{-1})}\norm{\Pi f}_{L^2(G^{-1})}
\end{align*}
and thanks to \eqref{eq:bdd2} we have
\begin{align*}
    \bangle{B^{-1}\Div_x m_f , \Div_x m_f}_{\sfL^2(e^V)}&=\bangle{ \gradx(B^{-1}(\Div_x m_f)e^V ), m_f}_{\sfL^2(dx)}\\
    &\leq\( \frac{1}{2}\int_\Rd \abs{ \gradx(B^{-1}(\Div_x m_f)e^V )}^2e^{-V}\)^{1/2}\(\int_\Rd \abs{m_f}^2e^V\)^{1/2}\\
    &\leq \frac{\sqrt{2}}{2}\int_\Rd \abs{m_f}^2e^V\\
    &\leq \frac{\sqrt{2}}{2}\norm{(1-\Pi)f}^2_{\sfL^2(G^{-1})}.
\end{align*}
Thus
\begin{equation}
    \boxed{\bangle{B^{-1}\Div_x m_f , \rho_f}_{\sfL^2(e^V)}\leq \frac12\norm{(1-\Pi)f}_{L^2(G^{-1})}\norm{\Pi f}_{L^2(G^{-1})}}
\end{equation}
and
\begin{equation}
    \boxed{\bangle{B^{-1}\Div_x m_f , \Div_x m_f}_{\sfL^2(e^V)}\leq \frac{\sqrt{2}}{2}\norm{(1-\Pi)f}^2_{\sfL^2(G^{-1})}.}
\end{equation}

\subsubsection{Elliptic regularity} For the remaining term we have
\begin{align*}
    \bangle{B^{-1} \Div_x(\Div_x E_f), \rho_f}_{\sfL^2(e^V)} = &\bangle{E_f, \gradx^2((B^*)^{-1}(\rho_f e^V))}_{\sfL^2(dx)}\\
    &\leq \norm{E_f}_{\sfL^2(e^V)} \norm{\gradx^2((B^*)^{-1}(\rho_f e^V))}_{\sfL^2(e^{-V})}\\
    &\leq C_4\norm{(1-\Pi)f}_{\sfL^2(G^{-1})}\norm{\gradx^2((B^*)^{-1}(\rho_f e^V))}_{\sfL^2(e^{-V})}
\end{align*}
where $B^*$ is the adjoint in $\sfL^2(dx)$ of $B$.


% REWRITE TRANSFORM_MATCHING_TEX BY FADING OUT WHAT DOES NOT HAVE A CORRESPONDENCE
